\documentclass[./RTPostureTrackingReport.tex]{subfiles}

\begin{document}
\chapter{Conclusion \& Future work}\label{chap:concl}
\section{Outcomes}\label{sec:conclout}
The key outcomes of this project are to demonstrate a simple and effective
solution to improve people's ergonomics, especially during a time when not
everyone has access to luxuries like standing desks, specialised chairs, etc.
The simplicity of this proof-of-concept is exemplified by the fact that there
are only two active edge nodes, and it's effectiveness is shown by the
reliability of the data collected. On the whole, it was a very enriching
experience. This provided us the opportunity to explore areas we were
previously unfamiliar with.

\section{Limitations and Improvements}\label{sec:concllim}
The work done so far to come up with a proof-of-concept was successful, however
there is a lot of room for improvement. A few limitations are:
\begin{itemize}
    \item Filtering of ``noisy'' movements that can cause false positives
    \item Choosing a proper cloud solution for deploying the database and
        web-app
\end{itemize}

Apart from addressing these limitations, a few other possible additions to this
project are:
\begin{itemize}
    \item Ability to detect a wider range of activities
    \item Eliminate the host system from the architecture and move its
        functionality to the edge node.
    \item Perform a more thorough analysis of the data and provide greater
        insights and diagnostics to the user.
\end{itemize}
\vspace{2cm}

\hrule
\vspace{2cm}

\centering
This project is hosted at: \url{https://github.com/skandaprasad/mpu6050-datalogger}
\\
\vspace{0.5cm}
This document is typeset using \LaTeX

\end{document}
