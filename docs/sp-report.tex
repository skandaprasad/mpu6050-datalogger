\documentclass{report}
\usepackage[utf8]{inputenc}
\usepackage{float}

\begin{document}

\chapter{Introduction}

Every person who has a desk job, unconsciously is seated in an inconvenient position for prolonged duration. In the long run, these seating postures might lead to undesirable health issues such as aches in the back and/or the neck.  
Hence, this project aims to observe the posture of a person and notify the duration of time for which the person was seated in a particular posture.
\\
This is achieved by:
\begin{itemize}
    \item Setting up a sensor-interfaced microcontroller to obtain the real-time data.
    \item Transmit the collected data points to a database.
    \item Classify the posture with the help of the data collected.
\end{itemize} \\
The following are the components used for data collection:
\begin{itemize}
    \item MPU6050 Gyro-Accelerometer
    \item Espressif's ESP8266 
    \item Arduino UNO Microcontroller
\end{itemize}

The data is also plotted in real-time for visualization purposes and finer interpretation. 

\chapter{Architecture}
\section{Espressif's ESP8266}

Espressif’s ESP8266 highly integrated Wi-Fi SoC solution for efficient power usage, compact design and reliable performance in the Internet of Things industry. The integrated high-speed cache helps to increase the system performance and optimize the system memory. \\ 
Most importantly, ESP8266 also integrates a 32-bit processor and on-chip SRAM.
It supports UART, SDIO, SPI, I2C, I2S, IR Remote Control protocols for communication purposes.
Also, it has 17 GPIO pins which can be assigned to various functions by programming the appropriate registers.

\section{Arduino UNO}

The Arduino Uno is a microcontroller board based on the ATmega328P. It has 14 digital input/output pins (of which 6 can be used as PWM outputs) and 6 analog inputs. \\ The ATmega328 has 32 KB (with 0.5 KB used for the bootloader). It also has 2 KB of SRAM and 1 KB of EEPROM. The ATmega328 also supports I2C (TWI) and SPI communication.

\section{MPU6050 Gyro-Accelerometer}

The MPU6050 contains a MEMS (Micro-electromechanical System) 3-axis gyroscope and a 3-axis accelerometer on the same silicon die together with an on-board Digital Motion Processor. \\ The MPU-60X0 has three 16-bit analog-to-digital converters (ADCs) each for digitizing the gyroscope outputs and the accelerometer outputs.
\newpage
For our application, the connections between MPU6050 Gyro-Accelerometer and the ESP8266 Microcontroller are as follows:
\begin{table}[H]
    \centering
    \begin{tabular}{|c|c|}
    \hline
    MPU6050 Gyro-Accelerometer & ESP8266 Microcontroller \\
    \hline
    VCC & VCC (3.3V) \\
    GND & GND \\
    SCL & D1 \\
    SDA & D2 \\
    \hline
\end{tabular}
    \caption{Pin connections between the sensor and the microcontroller}
    \label{tab:pin}
\end{table}

\end{document}
